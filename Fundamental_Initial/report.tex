\documentclass[12pt,a4paper,onecolumn]{article}
\usepackage[utf8]{inputenc}
\usepackage[T1]{fontenc}
\usepackage[french]{babel}

% ------------------------- Color table ----------------------------------------
\usepackage{multirow}
\usepackage[table]{xcolor}
\definecolor{maroon}{cmyk}{0,0.87,0.68,0.32}
% ------------------------------------------------------------------------------

\usepackage{amscd}
\usepackage{amsthm}
\usepackage{physics}
\usepackage[left=2.2cm,right=2.2cm,top=2cm,bottom=2cm]{geometry}
\usepackage{textcomp,gensymb} %pour le °C, et textcomp pour éviter les warning
\usepackage{graphicx} %pour les images
\usepackage{caption}
\usepackage{subcaption}
\usepackage[colorlinks=true,
	breaklinks=true,
	citecolor=blue,
	linkcolor=blue,
	urlcolor=blue]{hyperref} % pour insérer des liens
\usepackage{epstopdf} %converting to PDF
\usepackage[export]{adjustbox} %for large figures

\usepackage{array}
\usepackage{dsfont}% indicatrice : \mathds{1}


% -------------------------- Mathematics ---------------------------------------
\graphicspath{{images/}} % For the images path
% ------------------------------------------------------------------------------

% -------------------------- Mathematics ---------------------------------------
\usepackage{mathrsfs, amsmath, amsfonts, amssymb}
\usepackage{bm}
\usepackage[Symbol]{upgreek} % For pi \uppi different from /pi
\newcommand{\R}{\mathbb{R}} % For Real space
% ------------------------------------------------------------------------------


% -------------------------- Code format ---------------------------------------
\usepackage[numbered,framed]{matlab-prettifier}
\lstset{
	style              = Matlab-editor,
	basicstyle         = \mlttfamily,
	escapechar         = '',
	mlshowsectionrules = true,
}
% ------------------------------------------------------------------------------

% ------------------------- Blbiographie --------------------------------------
% \usepackage[backend=biber, style=science]{biblatex}
% \addbibresource{biblio.bib}
% ------------------------------------------------------------------------------


\setcounter{tocdepth}{4} %Count paragraph
\setcounter{secnumdepth}{4} %Count paragraph
\usepackage{float}

\usepackage{graphicx} % for graphicspath
% \graphicspath{{../images/}}

\usepackage{array,tabularx}
\newcolumntype{L}[1]{>{\raggedright\let\newline\\\arraybackslash\hspace{0pt}}m{#1}}
\newcolumntype{C}[1]{>{\centering\let\newline\\\arraybackslash\hspace{0pt}}m{#1}}
\newcolumntype{R}[1]{>{\raggedleft\let\newline\\\arraybackslash\hspace{0pt}}m{#1}}

\title{Matrice Fondamentale}
\author{Vincent Matthys}
\graphicspath{{images/}}


\renewcommand{\thesubsection}{\alph{subsection}}


\begin{document}
%\maketitle
\begin{tabularx}{\textwidth}{@{} l X r @{} }
	{\textsc{Master MVA}}          &  & \textsc{Scnenario d'usage} \\
	\textsc{UE 3D COMPUTER VISION} &  & {ENS Paris Saclay}         \\
	%& %M1 Informatique
\end{tabularx}
\vspace{1.5cm}
\begin{center}

	\rule[11pt]{5cm}{0.5pt}

	\textbf{\LARGE \textsc{Détermination de la matrice Fondamentale par algorithme ransac}}
	\vspace{0.5cm}

	Vincent Matthys


	\rule{5cm}{0.5pt}

	\vspace{1.5cm}
\end{center}

\section{Objectifs}
Le programme permet à l'utilisateur :
\begin{enumerate}
	\item de déterminer la matrice fondamentale à partir de deux images d'une même scène sans connaissance au préalable des paramètres internes de l'appareil
	\item déterminer les \textit{SIFT inliers}
	\item de tracer la ligne épipolaire associé à un point sélectionné par l'utilisateur dans l'autre image
\end{enumerate}


\section{Matériel \& méthodes}
Un seul fichier C++, nommé \textit{Fundamental.cpp} charge et affiche les deux images, qui peuvent être passés en paramètres du programme nommé \textit{Fundamental}. Un algorithme déterminant les SIFT est alors lancé sur chaque image, et les correspondances entre SIFT sont enregistrées.

Ces correspondances sont alors épurées par RANSAC, en calculant une matrice fondamentale avec l'algorithme des 8 points, et en discriminant les correspondances déterminées par SIFT matching en calculant la distance d'un point d'une correspondance à sa droite épipolaire associée. Si celle-ci est inférieure à \(10^{-5}\), alors la correspondance est considérée comme \textit{inlier}. Ce critère de distance a été retenu afin d'obtenir avec une certitude de \( \beta = 1~\%\) d'obtenir 8 correspondances non contaminées par des correspondances aberrantes par tirage aléatoire, en une trentaine d'itérations, ne gardant que \( 500\) correspondances sur les \( 675 \) retournées par SIFT matching.

L'algorithme des 8 points consiste à résoudre le système linéaire suivant :

\begin{equation*}
	\begin{split}
		A \dotproduct f = 0\\
		\begin{bmatrix}
			A_1^{\intercal} \\
			A_2^{\intercal} \\
			A_3^{\intercal} \\
			A_4^{\intercal} \\
			A_5^{\intercal} \\
			A_6^{\intercal} \\
			A_7^{\intercal} \\
			A_8^{\intercal} \\
		\end{bmatrix}
		\dotproduct
		\begin{bmatrix}
			f_{11} \\
			f_{12} \\
			f_{13} \\
			f_{21} \\
			f_{22} \\
			f_{23} \\
			f_{31} \\
			f_{32} \\
			f_{33}
		\end{bmatrix}
		&=
		0\\
		\begin{bmatrix}
			x_1x_1^{\prime} & x_1y_1^{\prime} & x_1 & y_1x_1^{\prime} & y_1y_1^{\prime} & y_1 & x_1^{\prime} & y_1^{\prime} & 1 \\
			x_2x_2^{\prime} & x_2y_2^{\prime} & x_2 & y_2x_2^{\prime} & y_2y_2^{\prime} & y_2 & x_2^{\prime} & y_2^{\prime} & 1 \\
			x_3x_3^{\prime} & x_3y_3^{\prime} & x_3 & y_3x_3^{\prime} & y_3y_3^{\prime} & y_3 & x_3^{\prime} & y_3^{\prime} & 1 \\
			x_4x_4^{\prime} & x_4y_4^{\prime} & x_4 & y_4x_4^{\prime} & y_4y_4^{\prime} & y_4 & x_4^{\prime} & y_4^{\prime} & 1 \\
			x_5x_5^{\prime} & x_5y_5^{\prime} & x_5 & y_5x_5^{\prime} & y_5y_5^{\prime} & y_5 & x_5^{\prime} & y_5^{\prime} & 1 \\
			x_6x_6^{\prime} & x_6y_6^{\prime} & x_6 & y_6x_6^{\prime} & y_6y_6^{\prime} & y_6 & x_6^{\prime} & y_6^{\prime} & 1 \\
			x_7x_7^{\prime} & x_7y_7^{\prime} & x_7 & y_7x_7^{\prime} & y_7y_7^{\prime} & y_7 & x_7^{\prime} & y_7^{\prime} & 1 \\
			x_8x_8^{\prime} & x_8y_8^{\prime} & x_8 & y_8x_8^{\prime} & y_8y_8^{\prime} & y_8 & x_8^{\prime} & y_8^{\prime} & 1
		\end{bmatrix}
		\dotproduct
		\begin{bmatrix}
			f_{11} \\
			f_{12} \\
			f_{13} \\
			f_{21} \\
			f_{22} \\
			f_{23} \\
			f_{31} \\
			f_{32} \\
			f_{33}
		\end{bmatrix}
		&=
		\begin{bmatrix}
			\vdots \\
			0      \\
			\vdots
		\end{bmatrix}
	\end{split}
\end{equation*}

Avec la contrainte \( \norm{f} = 1\). \( f\) est alors le dernier vecteur singulier à droite de la matrice \( A\). Le programme utilise une décomposition \(SVD\) carré en rajoutant l'équation \( 0 = 0\) comme dernière ligne de \(A\), ce qui implique alors le choix de l'avant dernier vecteur singulier à droite.

La matrice fondamentale est finalement calculée sur les \textit{inliers} restant déterminés par \textit{RANSAC} par minimization de l'erreur quadratique. du système précédemment présenté, avec \( f_{33} = 1\). Après décomposition \(SVD\), la dernière valeur singulière est mise à 0 pour forcer le rang de \( f\) à 2. \(f\) est ensuite recomposée.

Il faut noter que le calcul de \(f\) nécessite une normalization des échelles, qui est inversée après calcul. Cette normalization consiste à multiplier toutes de coordonnées par \( 10^{-3} \) pour ramener l'échelle autour de l'unité, pour des images de l'ordre du millier de pixels de dimension.

Enfin, pour tracer les lignes épipolaires associées au point sélectionné par l'utilisateur, cette même matrice fondamentale est utilisée ; par convention, la couleur jaune pour les points sélectionnés dans l'image de gauche et pour les lignes épipolaires associées dans l'image de droite sera utilisée, et la couleur rouge inversement pour l'autre image.

\section{Résultats}

En figure~\ref{fig:sift} est présenté les détecteurs SIFT pour chaque image, tels que détectés par le programme, au nombre de \( 1400 \) dans l'image de gauche et de \( 1395 \) dans l'image de droite, ramenés à \( 675 \) correspondances, par un test de similarité sur les descripteurs associés.

\begin{figure}[H]
	\centering
	\includegraphics[width = 0.95\textwidth]{6_3.png}
	\caption{SIFT détecteurs pour chacune des deux images : \( 1400 \) features détectés dans l'image de gauche, et \( 1395 \) features détectés dans l'image de droite}
	\label{fig:sift}
\end{figure}

De ces \( 675 \) correspondances, par \textit{RANSAC}, on garde \( 500 \) correspondances, après 48 itérations pour une matrice fondamentale, déterminée par minimization de l'erreur quadratique sur les correspondances sélectionnées suivante :

\[
	F=
	\begin{bmatrix}
		1.48~10^{-6}   & -7.41~10^{-5} & 1.60~10^{-2}  \\
		7.97~10^{-5}   & 9.48~10^{-6}  & -2.86~10^{-3} \\
		-1.181~10^{-2} & -3.72~10^{-3} & 0.988
	\end{bmatrix}
\]

En figure~\ref{fig:sift_inliers} sont présentés les \(500\) correspondances conservées après \textit{RANSAC}. On peut remarque que la totalité des détecteurs situés au niveau de l'herbe dans la partie inférieure de la scène ont disparu, ainsi que quelques détecteurs situés au niveau des schémas répétés du batiment central. Ces disparitions signifient un manque de robustesse de ces détecteurs, situés dans des zones répétées et donc non discernables localement.

\begin{figure}[H]
	\centering
	\includegraphics[width = 0.95\textwidth]{6_1.png}
	\caption{SIFT matching : les 500 inliers déterminés par \(F\), pour une distance à la droite épipolaire associée inférieure à \(10^{-5}\)}
	\label{fig:sift_inliers}
\end{figure}

Enfin en figure~\ref{fig:epipolar_lines} sont représentées quelques lignes épipolaires associés à quelques points sélectionnés à la main. Ces droites épipolaires se coupent à l'extérieur gauche de la première image, et tout au bord gauche de la deuxième image. Leur intersection définit l'épipôle de chaque image, qui se retrouve décalé sur la droite dans la deuxième image.

\begin{figure}[H]
	\centering
	\includegraphics[width = 0.95\textwidth]{6_2.png}
	\caption{Les droites épipolaires associées aux points sélectionnés, en jaune dans l'image de gauche, en rouge dans l'image de droite, concourrantes aux épipoles.}
	\label{fig:epipolar_lines}
\end{figure}

Il est à noter que les épipoles sont très sensibles à l'initialisation de \textit{RANSAC}. En effet, en figure~\ref{fig:sensitive}, sont présentés les correponsdances conservées par RANSAC, ainsi que les épipoles, qui se situent alors à l'extérieur à droite de chaque image, soit à l'opposé des épipoles déterminés précédemment en figure~\ref{fig:epipolar_lines}.

\begin{figure}[H]
	\centering
	\includegraphics[width = 0.95\textwidth]{7.png}
	\includegraphics[width = 0.95\textwidth]{7_1.png}
	\caption{Correspondaces SIFT conservées par RANSAC (520) et droites épipolaires, pour un autre tirage de \textit{RANSAC}}
	\label{fig:sensitive}
\end{figure}

Pour ce tirage, \( 520 \) correspondances ont été conservées pour déterminer la matrice fondamentale \( F_2 \) suivante en \( 34 \) itérations :

\[
	F_2=
	\begin{bmatrix}
		4.73~10^{-7}  & -3.34~10^{-5} & 3.73~10^{-3}  \\
		3.41~10^{-5}  & 1.11~10^{-5}  & -5.73~10^{-2} \\
		-4.93~10^{-3} & 5.25~10^{-2}  & 1.00
	\end{bmatrix}
\]


\end{document}
