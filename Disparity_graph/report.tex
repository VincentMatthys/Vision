\documentclass[12pt,a4paper,onecolumn]{article}
\usepackage[utf8]{inputenc}
\usepackage[T1]{fontenc}
\usepackage[french]{babel}

% ------------------------- Color table ----------------------------------------
\usepackage{multirow}
\usepackage[table]{xcolor}
\definecolor{maroon}{cmyk}{0,0.87,0.68,0.32}
% ------------------------------------------------------------------------------

\usepackage{amscd}
\usepackage{amsthm}
\usepackage{physics}
\usepackage[left=2.2cm,right=2.2cm,top=2cm,bottom=2cm]{geometry}
\usepackage{textcomp,gensymb} %pour le °C, et textcomp pour éviter les warning
\usepackage{graphicx} %pour les images
\usepackage{caption}
\usepackage{subcaption}
\usepackage[colorlinks=true,
	breaklinks=true,
	citecolor=blue,
	linkcolor=blue,
	urlcolor=blue]{hyperref} % pour insérer des liens
\usepackage{epstopdf} %converting to PDF
\usepackage[export]{adjustbox} %for large figures

\usepackage{array}
\usepackage{dsfont}% indicatrice : \mathds{1}


% -------------------------- Mathematics ---------------------------------------
\graphicspath{{images/}{../images/}} % For the images path
% ------------------------------------------------------------------------------

% -------------------------- Mathematics ---------------------------------------
\usepackage{mathrsfs, amsmath, amsfonts, amssymb}
\usepackage{bm}
\usepackage{mathtools}
\usepackage[Symbol]{upgreek} % For pi \uppi different from /pi
\newcommand{\R}{\mathbb{R}} % For Real space

% ------------------------------------------------------------------------------


% -------------------------- Code format ---------------------------------------
\usepackage[numbered,framed]{matlab-prettifier}
\lstset{
	style              = Matlab-editor,
	basicstyle         = \mlttfamily,
	escapechar         = '',
	mlshowsectionrules = true,
}
% ------------------------------------------------------------------------------

% ------------------------- Blbiographie --------------------------------------
% \usepackage[backend=biber, style=science]{biblatex}
% \addbibresource{biblio.bib}
% ------------------------------------------------------------------------------


\setcounter{tocdepth}{4} %Count paragraph
\setcounter{secnumdepth}{4} %Count paragraph
\usepackage{float}

\usepackage{graphicx} % for graphicspath
% \graphicspath{{../images/}}

\usepackage{array,tabularx}
\newcolumntype{L}[1]{>{\raggedright\let\newline\\\arraybackslash\hspace{0pt}}m{#1}}
\newcolumntype{C}[1]{>{\centering\let\newline\\\arraybackslash\hspace{0pt}}m{#1}}
\newcolumntype{R}[1]{>{\raggedleft\let\newline\\\arraybackslash\hspace{0pt}}m{#1}}

% to start counting section to 6


% ------------------------ General informations --------------------------------
\title{MVA - 3D Computer Vision - Disparity using graph cuts}
\author{Vincent Matthys}
\graphicspath{{images/}{../images/}} % For the images path
% ------------------------------------------------------------------------------


\begin{document}

\begin{tabularx}{0.9\textwidth}{@{} l X r @{} }
	{\textsc{Master MVA}}       &  & \textsc{Vincent Matthys} \\
	\textsc{3D Computer vision} &  & {ENS Paris Saclay}       \\
\end{tabularx}
\vspace{1.5cm}
\begin{center}

	\rule[11pt]{5cm}{0.5pt}

	\textbf{\LARGE \textsc{Disparity map estimation using graph cuts}}
	\vspace{0.5cm}

	Vincent Matthys

	vincent.matthys@ens-paris-saclay.fr

	\rule{5cm}{0.5pt}

	\vspace{1.5cm}
\end{center}

\tableofcontents

\clearpage

Ce rapporte accompagne le programme \textit{stereoGC} d'estimation de carte de disparité à partir de deux vues d'une même scène. Afin d'executer ce programme, il faut dans un premier temps lancer les commandes suivantes

\begin{lstlisting}[numbers = none]
cmake CMakeLists.txt
make
\end{lstlisting}

Qui produit un executable dans le répertoire courant \textit{stereoGC}. On exécute le programme sur deux images \textit{im1} et \textit{im2} de la façon suivante :

\begin{lstlisting}[numbers = none]
./stereoGC im1 im2
\end{lstlisting}

Deux utilisations modèles sont utilisables en définissant dans le préambule le \textit{define} correspondants :
\begin{itemize}
	\item THIERRY : charge les deux images du visage de Thierry ainsi que les paramètres raisonnables associés ; en particulier on se limite à une recherche de disparité \(d \in [10, 55]\).
	\item TOYS : charge les deux images de la scène des jouets ainsi que les paramètre raisonnables associés  ; en particulier on se limite à une recherche de disparité \(d \in [-30, -7]\).
\end{itemize}

Le rapport qui suit regroupe différentes observations quand à son utilisation sur différentes images et avec différents paramètres.

\section{Présentation de cas fonctionnels}


\section{Comparaison avec l'expansion de graines}


\end{document}
