\documentclass[12pt,a4paper,onecolumn]{article}
\input{packages}
\input{macros}

% ------------------------ General informations --------------------------------
\title{MVA - 3D Computer Vision - Disparity using graph cuts}
\author{Vincent Matthys}
\graphicspath{{images/}{../images/}} % For the images path
% ------------------------------------------------------------------------------


\begin{document}

\begin{tabularx}{0.9\textwidth}{@{} l X r @{} }
	{\textsc{Master MVA}}       &  & \textsc{Vincent Matthys} \\
	\textsc{3D Computer vision} &  & {ENS Paris Saclay}       \\
\end{tabularx}
\vspace{1.5cm}
\begin{center}

	\rule[11pt]{5cm}{0.5pt}

	\textbf{\LARGE \textsc{Disparity map estimation using graph cuts}}
	\vspace{0.5cm}

	Vincent Matthys

	vincent.matthys@ens-paris-saclay.fr

	\rule{5cm}{0.5pt}

	\vspace{1.5cm}
\end{center}

\tableofcontents

\clearpage

Ce rapporte accompagne le programme \textit{stereoGC} d'estimation de carte de disparité à partir de deux vues d'une même scène. Afin d'executer ce programme, il faut dans un premier temps lancer les commandes suivantes

\begin{lstlisting}[numbers = none]
cmake CMakeLists.txt
make
\end{lstlisting}

Qui produit un executable dans le répertoire courant \textit{stereoGC}. On exécute le programme sur deux images \textit{im1} et \textit{im2} de la façon suivante :

\begin{lstlisting}[numbers = none]
./stereoGC im1 im2
\end{lstlisting}

Deux utilisations modèles sont utilisables en définissant dans le préambule le \textit{define} correspondants :
\begin{itemize}
	\item THIERRY : charge les deux images du visage de Thierry ainsi que les paramètres raisonnables associés ; en particulier on se limite à une recherche de disparité \(d \in [10, 55]\).
	\item TOYS : charge les deux images de la scène des jouets ainsi que les paramètre raisonnables associés  ; en particulier on se limite à une recherche de disparité \(d \in [-30, -7]\).
\end{itemize}

Le rapport qui suit regroupe différentes observations quand à son utilisation sur différentes images et avec différents paramètres.

\section{Présentation de cas fonctionnels}

\subsection{Influence de la valeur du poids entre pixel non corrélés}

En figure~\ref{fig_thierry_wcc} est présenté l'influence du poids \(w_{cc}\) entre pixels non corrélés sur la carte de disparité. Théoriquement, ce poids influence le niveau de discrétisation de l'énergie du graphe. En pratique, comme visible en figure~\ref{fig_thierry_disp_wcc_1000}, si ce poids est trop élevé, alors les variations de disparités entraînent une grande variation d'énergie, que le lissage ne suffit pas à compenser.

\begin{figure}[H]
	\begin{subfigure}[b]{0.3\textwidth}
		\centering
		\includegraphics[height = 0.2\textheight, angle = -90]{thierry_disp_blur_3_z_2}
		\caption{\(w_{cc} = 20\)}
		\label{fig_thierry_disp_wcc_20}
	\end{subfigure}
	\hfill
	\begin{subfigure}[b]{0.3\textwidth}
		\centering
		\includegraphics[height = 0.2\textheight, angle = -90]{thierry_disp_wcc_100}
		\caption{\(w_{cc} = 100\)}
		\label{fig_thierry_disp_wcc_100}
	\end{subfigure}
	\hfill
	\begin{subfigure}[b]{0.3\textwidth}
		\centering
		\includegraphics[height = 0.2\textheight, angle = -90]{thierry_disp_wcc_1000}
		\caption{\(w_{cc} = 1000\)}
		\label{fig_thierry_disp_wcc_1000}
	\end{subfigure}
	\hfill
	\begin{subfigure}[b]{0.3\textwidth}
		\centering
		\includegraphics[height = 0.2\textheight]{thierry_3D_blur_3_z_2}
		\caption{\(w_{cc} = 20\)}
		\label{fig_thierry_3D_wcc_20}
	\end{subfigure}
	\hfill
	\begin{subfigure}[b]{0.3\textwidth}
		\centering
		\includegraphics[height = 0.2\textheight]{thierry_3D_wcc_100}
		\caption{\(w_{cc} = 100\)}
		\label{fig_thierry_3D_wcc_100}
	\end{subfigure}
	\hfill
	\begin{subfigure}[b]{0.3\textwidth}
		\centering
		\includegraphics[height = 0.2\textheight]{thierry_3D_wcc_1000}
		\caption{\(w_{cc} = 1000\)}
		\label{fig_thierry_3D_wcc_1000}
	\end{subfigure}
	\hfill
	\caption{Carte de disparité obtenue pour différentes valeurs de poids entre pixels non corrélés \(w_{cc}\)}
	\label{fig_thierry_wcc}
\end{figure}


\subsection{Influence de la valeur de lissage}

En figure~\ref{fig_thierry} est présenté le résultat obtenu pour différentes valeurs de lissage. On constate que les valeurs 3, 5 et 8 donnent des résultats similaires pour cette reconstruction. En revanche, comme attendu, des valeurs trop faibles de \(\sigma\) donne des discontinuités omniprésentes. A l'inverse, une trop grande valeur de lissage fait perdre de l'information.

\begin{figure}[H]
	\centering
	\begin{subfigure}[b]{0.3\textwidth}
		\centering
		\includegraphics[height = 0.2\textheight, angle = -90]{thierry_disp_blur_1_z_2}
		\caption{\(\sigma = 1\)}
		\label{fig_thierry_disp_1}
	\end{subfigure}
	\hfill
	\begin{subfigure}[b]{0.3\textwidth}
		\centering
		\includegraphics[height = 0.2\textheight, angle = -90]{thierry_disp_blur_2_z_2}
		\caption{\(\sigma = 2\)}
		\label{fig_thierry_disp_2}
	\end{subfigure}
	\hfill
	\begin{subfigure}[b]{0.3\textwidth}
		\centering
		\includegraphics[height = 0.2\textheight, angle = -90]{thierry_disp_blur_3_z_2}
		\caption{\(\sigma = 3\)}
		\label{fig_thierry_disp_3}
	\end{subfigure}
	\hfill
	\begin{subfigure}[b]{0.3\textwidth}
		\centering
		\includegraphics[height = 0.2\textheight]{thierry_3D_blur_1_z_2}
		\caption{\(\sigma = 1\)}
		\label{fig_thierry_3D_1}
	\end{subfigure}
	\hfill
	\begin{subfigure}[b]{0.3\textwidth}
		\centering
		\includegraphics[height = 0.2\textheight]{thierry_3D_blur_2_z_2}
		\caption{\(\sigma = 2\)}
		\label{fig_thierry_3D_1}
	\end{subfigure}
	\hfill
	\begin{subfigure}[b]{0.3\textwidth}
		\centering
		\includegraphics[height = 0.2\textheight]{thierry_3D_blur_3_z_2}
		\caption{\(\sigma = 3\)}
		\label{fig_thierry_3D_1}
	\end{subfigure}
	\hfill
	\begin{subfigure}[b]{0.3\textwidth}
		\centering
		\includegraphics[height = 0.2\textheight, angle = -90]{thierry_disp_blur_5_z_2}
		\caption{\(\sigma = 5\)}
		\label{fig_thierry_disp_5}
	\end{subfigure}
	\hfill
	\begin{subfigure}[b]{0.3\textwidth}
		\centering
		\includegraphics[height = 0.2\textheight, angle = -90]{thierry_disp_blur_8_z_2}
		\caption{\(\sigma = 8\)}
		\label{fig_thierry_disp_8}
	\end{subfigure}
	\hfill
	\begin{subfigure}[b]{0.3\textwidth}
		\centering
		\includegraphics[height = 0.2\textheight, angle = -90]{thierry_disp_blur_15_z_2}
		\caption{\(\sigma = 15\)}
		\label{fig_thierry_disp_15}
	\end{subfigure}
	\hfill
	\begin{subfigure}[b]{0.3\textwidth}
		\centering
		\includegraphics[height = 0.2\textheight]{thierry_3D_blur_5_z_2}
		\caption{\(\sigma = 5\)}
		\label{fig_thierry_3D_5}
	\end{subfigure}
	\hfill
	\begin{subfigure}[b]{0.3\textwidth}
		\centering
		\includegraphics[height = 0.2\textheight]{thierry_3D_blur_8_z_2}
		\caption{\(\sigma = 8\)}
		\label{fig_thierry_3D_8}
	\end{subfigure}
	\hfill
	\begin{subfigure}[b]{0.3\textwidth}
		\centering
		\includegraphics[height = 0.2\textheight]{thierry_3D_blur_15_z_2}
		\caption{\(\sigma = 15\)}
		\label{fig_thierry_3D_15}
	\end{subfigure}
	\caption{Carte de disparité obtenue pour différentes valeurs de lissage \(\sigma\)}
	\label{fig_thierry}
\end{figure}


\section{Comparaison avec l'expansion de graines}


\end{document}
