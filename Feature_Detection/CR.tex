\documentclass[12pt,a4paper,onecolumn]{article}
\input{packages}
\input{macros}

% ------------------------ General informations --------------------------------
\title{Math M2 Probabilistic graphical models 2017/2018}
\author{Vincent Matthys}
\graphicspath{{images/}}
% ------------------------------------------------------------------------------

\begin{document}
\begin{tabularx}{0.8\textwidth}{@{} l X r @{} }
	{\textsc{Master MVA}}       &  & \textsc{Rapport expérimental} \\
	\textsc{3D Computer Vision} &  & {Vincent Matthys}             \\
\end{tabularx}
\vspace{1.5cm}
\begin{center}
	\rule[11pt]{5cm}{0.5pt}

	\textbf{\LARGE \textsc{Détecteurs de coins d'Harris}}
	\vspace{0.5cm}\\
	Vincent Matthys\\
	\rule{5cm}{0.5pt}
	\vspace{1.5cm}
\end{center}

Ce projet consiste en un développement dans l'élaborateur d'un détecteur de coins d'Harris, ainsi que d'un rafinement des détections par \textit{adaptive non-maximal suppression}. Le rapport qui suit regroupe différentes observations quand à son utilisation sur différentes images, avec différents paramètres et dans des conditions différentes, ceci dans un soucis d'évaluation qualitative du progamme délivré.

\tableofcontents
\section{Différentes scènes}

En figure~\ref{fig_1} sont présentés 4 résultats donnés par le détecteurs de coins de Harris implémenté, sur des images de taille comparable, avec les paramètres suivants :
\begin{itemize}
	\item \( \sigma_d = 1\) écart-type du noyau de convolution pour la dérivation.
	\item \( \sigma_i = 1\) écart-type de la fenêtre gaussienne de lissage des images produits.
	\item \( \kappa = 0.05\) constante multiplicative de la fonction de réponse
	\item \( threshold = 0.01\) seuil global (en fonction de la réponse maximale) en deça duquel aucune réponse n'est considérée.
	\item \(local = 3\) délimite la fenêtre de recherche du maximum local
	\item \(c = 0.7\) seuillage au delà duquel le point est automatiquement détecté.
	\item \( b = 50\) nombre de détections souhaité.
\end{itemize}
\begin{figure}[H]
	\centering
	\begin{subfigure}[b]{\textwidth}
		\centering
		\includegraphics[height = 0.20\textheight]{1_bouc}
		\subcaption{Image Bouc, 495x495 pixels : 1265 coins détectés avant anms, 50 après}
		\label{fig_1_bouc}
	\end{subfigure}
	\begin{subfigure}[b]{\textwidth}
		\centering
		\includegraphics[height = 0.20\textheight]{1_cameraman}
		\subcaption{Image Cameraman, 256x256 pixels : 156 coins détectés avant anms, 50 après}
		\label{fig_1_cameraman}
	\end{subfigure}
	\begin{subfigure}[b]{\textwidth}
		\centering
		\includegraphics[height = 0.20\textheight]{1_lena}
		\subcaption{Image Lena, 512x512 pixels : 326 coins détectés avant anms, 50 après}
		\label{fig_1_lena}
	\end{subfigure}
	\begin{subfigure}[b]{\textwidth}
		\centering
		\includegraphics[height = 0.20\textheight]{1_room}
		\subcaption{Image Room, 512x512 pixels : 407 coins détectés avant anms, 50 après}
		\label{fig_1_room}
	\end{subfigure}
	\caption{Détecteurs de coins d'Harris sur différentes scènes. Colonne de gauche, sans \textit{adaptative non-maximal supression} ; colonne de droite, avec \textit{adaptative non-maximal supression}}
	\label{fig_1}
\end{figure}
En figure~\ref{fig_1_bouc}, on constate un très grand nombre de coins détectés, 5 fois supérieur aux autres images de même taille. Ceci est très bien expliqué par la géométrie du circuit représenté dans l'image, composé essentiellement de lignes ayant une forte réponse dans les images produits. Dans les 3 autres images~\ref{fig_1_cameraman}~\ref{fig_1_lena}~\ref{fig_1_room}, le nombre de coins détectés est sensiblement identique, si on le raméne à la taille de l'image. On constate légèrement plus de détections dans l'image Room pour les mêmes raisons que l'image Bouc, dans le carré de tissu présentant des lignes non totalement horizontales répondant fortement dans les images produits.

Une autre constatation importante est la localisation des détections, qui se situent dans les zones non constantes par morceaux, \textit{a fortiori} très constratées. Ceci est encore une fois expliqué par la réponse nulle de telles zones dans les images produits.

Enfin, l'\textit{adaptative non-maximal supression} agit comme attendu, gardant des détections de façon non homogène, dépendante de la densité initiale mais en gardant un support spatial des détections semblable au support spatial avant \textit{adaptative non-maximal supression}.

\section{Différents paramètres}

\begin{table}[H]
	\centering
	\begin{tabular}{L{2cm} | C{2cm} | R{10cm}}
		\hline
		paramètres     & valeur par défaut & fonction                                                                                          \\\hline
		\( \sigma_d\)  & 1                 & écart-type du noyau de convolution pour la dérivation                                             \\\hline
		\( \sigma_i \) & 1                 & écart-type de la fenêtre gaussienne de lissage des images produits.                               \\\hline
		\( \kappa\)    & 0.05              & constante multiplicative de la fonction de réponse                                                \\\hline
		\( threshold\) & 0.01              & seuil global (en fonction de la réponse maximale) en deça duquel aucune réponse n'est considérée. \\\hline
		\(local \)     & 3                 & délimite la fenêtre de recherche du maximum local                                                 \\\hline
		\(c \)         & 0.7               & seuillage au delà duquel le point est automatiquement détecté.                                    \\\hline
		\( b\)         & 50                & nombre de détections souhaité.                                                                    \\\hline
	\end{tabular}
	\caption{Ensemble des paramètres du détecteur}
	\label{table_1}
\end{table}

Les expériences suivantes ont pour but d'étudier l'influence de chaque paramètre sur les détections. Aussi, les paramètres, sauf si précisés, seront ceux par défaut, présentés en table~\ref{table_1}.

\subsection{Influence de \(\sigma_d\)}

L'influence de \(\sigma_d\) est illustrée en figure~\ref{fig_2_sigma_d}, où le détecteur a été utlisé sur l'image Cameraman sans \textit{anms}. On s'apperçoit d'une part que le nombre de détection diminue puis augmente avec \(\sigma_d\). La diminution est due à l'élargissement de la zone sur laquelle la dérivée est calculée. L'image étant majoritairement constante par morceaux, une augmentation de \(\sigma_d\) entraîne une diminution des variations sur la fenêtre de calcul. A l'inverse, si \(\sigma_d\) est trop élevé, les effets de bords prennent le dessus, et le détecteur calcule une dérivée entre une partie de l'image, non noire, et une autre partie de l'image, complétée par 0-padding.
Il est surtout intéressant de noter la répartition de la disparition des détections le long des bords et coins fins de l'image, comme la face de la caméra, le manche, et les bords du trépied, entre \(\sigma_d = 1\) et \(\sigma_d = 5\).

\begin{figure}[H]
	\centering
	\includegraphics[width = 1.0\textwidth]{2_cameraman_sd}
	\caption{Influence de \(\sigma_d\) sur les détections, effectuées sans \textit{anms}}
	\label{fig_2_sigma_d}
\end{figure}

\subsection{Influence de \(\sigma_i\)}

L'influence de \(\sigma_i\) est illustrée en figure~\ref{fig_2_sigma_i}, où le détecteur a été utlisé sur l'image Cameraman sans \textit{anms}. Les observations et les conclusions sont les mêmes que pour \(\sigma_d\), si ce n'est que l'échelle de validité de \(\sigma_i\) est raccourcie : après \(\sigma_i = 5\), on semble perd d'informations.

\begin{figure}[H]
	\centering
	\includegraphics[width = 1.0\textwidth]{2_cameraman_si}
	\caption{Influence de \(\sigma_i\) sur les détections, effectuées sans \textit{anms}}
	\label{fig_2_sigma_i}
\end{figure}

\subsection{Influence de \(\kappa\)}

L'influence de \(\kappa\) est illustrée en figure~\ref{fig_2_kappa}, où le détecteur a été utlisé sur l'image Cameraman sans \textit{anms}. Théoriquement, \(\kappa\) est utilisé pour avoir une réponse positive si les deux valeurs propres de la matrix d'auto-corrélation sont "grandes". On constate que pour \(\kappa \in [0, 0.08]\), les détections ne changent ne varient ni en nombre, ni en localisation. Par contre après 0.08, les détections sont d'autant moins nombreuses que \(\kappa\) augmente, et disparaissent totalement après \(\kappa = 0.28\). Cette limite varie en fonction de l'image, contrairement à la première zone de relative stabilité. Il est donc raisonnable de travailler avec \(\kappa \in [0, 0.08]\) par défaut, et on prendre \(\kappa = 0.05 \) pour le reste des expériences.

\begin{figure}[H]
	\centering
	\includegraphics[width = 1.0\textwidth]{2_cameraman_kappa}
	\caption{Influence de \(\kappa\) sur les détections, effectuées sans \textit{anms}}
	\label{fig_2_kappa}
\end{figure}

\end{document}
